% Options for packages loaded elsewhere
\PassOptionsToPackage{unicode}{hyperref}
\PassOptionsToPackage{hyphens}{url}
%
\documentclass[
]{article}
\usepackage{lmodern}
\usepackage{amsmath}
\usepackage{ifxetex,ifluatex}
\ifnum 0\ifxetex 1\fi\ifluatex 1\fi=0 % if pdftex
  \usepackage[T1]{fontenc}
  \usepackage[utf8]{inputenc}
  \usepackage{textcomp} % provide euro and other symbols
  \usepackage{amssymb}
\else % if luatex or xetex
  \usepackage{unicode-math}
  \defaultfontfeatures{Scale=MatchLowercase}
  \defaultfontfeatures[\rmfamily]{Ligatures=TeX,Scale=1}
\fi
% Use upquote if available, for straight quotes in verbatim environments
\IfFileExists{upquote.sty}{\usepackage{upquote}}{}
\IfFileExists{microtype.sty}{% use microtype if available
  \usepackage[]{microtype}
  \UseMicrotypeSet[protrusion]{basicmath} % disable protrusion for tt fonts
}{}
\makeatletter
\@ifundefined{KOMAClassName}{% if non-KOMA class
  \IfFileExists{parskip.sty}{%
    \usepackage{parskip}
  }{% else
    \setlength{\parindent}{0pt}
    \setlength{\parskip}{6pt plus 2pt minus 1pt}}
}{% if KOMA class
  \KOMAoptions{parskip=half}}
\makeatother
\usepackage{xcolor}
\IfFileExists{xurl.sty}{\usepackage{xurl}}{} % add URL line breaks if available
\IfFileExists{bookmark.sty}{\usepackage{bookmark}}{\usepackage{hyperref}}
\hypersetup{
  pdftitle={Trabajo Práctico Probabilidades},
  pdfauthor={Federico Brusa - Javier Garcia Skabar},
  hidelinks,
  pdfcreator={LaTeX via pandoc}}
\urlstyle{same} % disable monospaced font for URLs
\usepackage[margin=1in]{geometry}
\usepackage{color}
\usepackage{fancyvrb}
\newcommand{\VerbBar}{|}
\newcommand{\VERB}{\Verb[commandchars=\\\{\}]}
\DefineVerbatimEnvironment{Highlighting}{Verbatim}{commandchars=\\\{\}}
% Add ',fontsize=\small' for more characters per line
\usepackage{framed}
\definecolor{shadecolor}{RGB}{248,248,248}
\newenvironment{Shaded}{\begin{snugshade}}{\end{snugshade}}
\newcommand{\AlertTok}[1]{\textcolor[rgb]{0.94,0.16,0.16}{#1}}
\newcommand{\AnnotationTok}[1]{\textcolor[rgb]{0.56,0.35,0.01}{\textbf{\textit{#1}}}}
\newcommand{\AttributeTok}[1]{\textcolor[rgb]{0.77,0.63,0.00}{#1}}
\newcommand{\BaseNTok}[1]{\textcolor[rgb]{0.00,0.00,0.81}{#1}}
\newcommand{\BuiltInTok}[1]{#1}
\newcommand{\CharTok}[1]{\textcolor[rgb]{0.31,0.60,0.02}{#1}}
\newcommand{\CommentTok}[1]{\textcolor[rgb]{0.56,0.35,0.01}{\textit{#1}}}
\newcommand{\CommentVarTok}[1]{\textcolor[rgb]{0.56,0.35,0.01}{\textbf{\textit{#1}}}}
\newcommand{\ConstantTok}[1]{\textcolor[rgb]{0.00,0.00,0.00}{#1}}
\newcommand{\ControlFlowTok}[1]{\textcolor[rgb]{0.13,0.29,0.53}{\textbf{#1}}}
\newcommand{\DataTypeTok}[1]{\textcolor[rgb]{0.13,0.29,0.53}{#1}}
\newcommand{\DecValTok}[1]{\textcolor[rgb]{0.00,0.00,0.81}{#1}}
\newcommand{\DocumentationTok}[1]{\textcolor[rgb]{0.56,0.35,0.01}{\textbf{\textit{#1}}}}
\newcommand{\ErrorTok}[1]{\textcolor[rgb]{0.64,0.00,0.00}{\textbf{#1}}}
\newcommand{\ExtensionTok}[1]{#1}
\newcommand{\FloatTok}[1]{\textcolor[rgb]{0.00,0.00,0.81}{#1}}
\newcommand{\FunctionTok}[1]{\textcolor[rgb]{0.00,0.00,0.00}{#1}}
\newcommand{\ImportTok}[1]{#1}
\newcommand{\InformationTok}[1]{\textcolor[rgb]{0.56,0.35,0.01}{\textbf{\textit{#1}}}}
\newcommand{\KeywordTok}[1]{\textcolor[rgb]{0.13,0.29,0.53}{\textbf{#1}}}
\newcommand{\NormalTok}[1]{#1}
\newcommand{\OperatorTok}[1]{\textcolor[rgb]{0.81,0.36,0.00}{\textbf{#1}}}
\newcommand{\OtherTok}[1]{\textcolor[rgb]{0.56,0.35,0.01}{#1}}
\newcommand{\PreprocessorTok}[1]{\textcolor[rgb]{0.56,0.35,0.01}{\textit{#1}}}
\newcommand{\RegionMarkerTok}[1]{#1}
\newcommand{\SpecialCharTok}[1]{\textcolor[rgb]{0.00,0.00,0.00}{#1}}
\newcommand{\SpecialStringTok}[1]{\textcolor[rgb]{0.31,0.60,0.02}{#1}}
\newcommand{\StringTok}[1]{\textcolor[rgb]{0.31,0.60,0.02}{#1}}
\newcommand{\VariableTok}[1]{\textcolor[rgb]{0.00,0.00,0.00}{#1}}
\newcommand{\VerbatimStringTok}[1]{\textcolor[rgb]{0.31,0.60,0.02}{#1}}
\newcommand{\WarningTok}[1]{\textcolor[rgb]{0.56,0.35,0.01}{\textbf{\textit{#1}}}}
\usepackage{graphicx}
\makeatletter
\def\maxwidth{\ifdim\Gin@nat@width>\linewidth\linewidth\else\Gin@nat@width\fi}
\def\maxheight{\ifdim\Gin@nat@height>\textheight\textheight\else\Gin@nat@height\fi}
\makeatother
% Scale images if necessary, so that they will not overflow the page
% margins by default, and it is still possible to overwrite the defaults
% using explicit options in \includegraphics[width, height, ...]{}
\setkeys{Gin}{width=\maxwidth,height=\maxheight,keepaspectratio}
% Set default figure placement to htbp
\makeatletter
\def\fps@figure{htbp}
\makeatother
\setlength{\emergencystretch}{3em} % prevent overfull lines
\providecommand{\tightlist}{%
  \setlength{\itemsep}{0pt}\setlength{\parskip}{0pt}}
\setcounter{secnumdepth}{-\maxdimen} % remove section numbering
\ifluatex
  \usepackage{selnolig}  % disable illegal ligatures
\fi

\title{Trabajo Práctico Probabilidades}
\author{Federico Brusa - Javier Garcia Skabar}
\date{5/09/2021}

\begin{document}
\maketitle

\hypertarget{ejercicio-1}{%
\subsubsection{Ejercicio 1)}\label{ejercicio-1}}

En una urna hay 4 bolas verdes, 3 amarillas y 3 rojas. Se extraen tres
bolas al azar sin reposición. Sean X la cantidad de bolas verdes e Y la
cantidad de bolas rojas extraídas.

\hypertarget{inciso-a}{%
\subparagraph{Inciso a)}\label{inciso-a}}

Simular 1000 realizaciones del experimento que consiste en extraer 3
bolas y observar el color, guardando el resultado de la cantidad de
verdes en el vector x, y la cantidad de rojas en el vector y.

\begin{Shaded}
\begin{Highlighting}[]
\NormalTok{urna }\OtherTok{\textless{}{-}} \FunctionTok{c}\NormalTok{(}\StringTok{"V"}\NormalTok{,}\StringTok{"V"}\NormalTok{,}\StringTok{"V"}\NormalTok{,}\StringTok{"V"}\NormalTok{,}\StringTok{"A"}\NormalTok{,}\StringTok{"A"}\NormalTok{,}\StringTok{"A"}\NormalTok{,}\StringTok{"R"}\NormalTok{,}\StringTok{"R"}\NormalTok{,}\StringTok{"R"}\NormalTok{)}
\NormalTok{resultados }\OtherTok{\textless{}{-}} \FunctionTok{data.frame}\NormalTok{(}\AttributeTok{ensayo =} \FunctionTok{seq}\NormalTok{(}\DecValTok{1}\NormalTok{,}\DecValTok{1000}\NormalTok{),}
                         \AttributeTok{X=}\FunctionTok{rep}\NormalTok{(}\ConstantTok{NA}\NormalTok{,}\DecValTok{1000}\NormalTok{),}
                         \AttributeTok{Y=}\FunctionTok{rep}\NormalTok{(}\ConstantTok{NA}\NormalTok{,}\DecValTok{1000}\NormalTok{))}
\FunctionTok{set.seed}\NormalTok{(}\DecValTok{17}\NormalTok{)}
\ControlFlowTok{for}\NormalTok{(i }\ControlFlowTok{in} \DecValTok{1}\SpecialCharTok{:}\FunctionTok{nrow}\NormalTok{(resultados))\{}
\NormalTok{  ensayo\_i }\OtherTok{\textless{}{-}} \FunctionTok{sample}\NormalTok{(}\DecValTok{1}\SpecialCharTok{:}\DecValTok{10}\NormalTok{, }\DecValTok{3}\NormalTok{, }\AttributeTok{replace=}\NormalTok{F)}
\NormalTok{  bolas }\OtherTok{\textless{}{-}}\NormalTok{ urna[ensayo\_i]}
\NormalTok{  resultados}\SpecialCharTok{$}\NormalTok{X[i] }\OtherTok{\textless{}{-}} \FunctionTok{sum}\NormalTok{(bolas}\SpecialCharTok{==}\StringTok{"V"}\NormalTok{)}
\NormalTok{  resultados}\SpecialCharTok{$}\NormalTok{Y[i] }\OtherTok{\textless{}{-}} \FunctionTok{sum}\NormalTok{(bolas}\SpecialCharTok{==}\StringTok{"R"}\NormalTok{)}
\NormalTok{\}}
\end{Highlighting}
\end{Shaded}

\hypertarget{inciso-b}{%
\subparagraph{Inciso b)}\label{inciso-b}}

Realizar un gráfico de puntos de x vs.~y. Qué se observa en este
gráfico?

\begin{Shaded}
\begin{Highlighting}[]
\FunctionTok{plot}\NormalTok{(resultados}\SpecialCharTok{$}\NormalTok{X, resultados}\SpecialCharTok{$}\NormalTok{Y, }\AttributeTok{xlab=}\StringTok{"X cantidad verdes"}\NormalTok{, }\AttributeTok{ylab=}\StringTok{"Y cantidad rojas"}\NormalTok{, }\AttributeTok{xaxt=}\StringTok{"n"}\NormalTok{, }\AttributeTok{yaxt=}\StringTok{"n"}\NormalTok{, }\AttributeTok{type=}\StringTok{"p"}\NormalTok{)}
\FunctionTok{axis}\NormalTok{(}\DecValTok{1}\NormalTok{, }\AttributeTok{at =} \FunctionTok{c}\NormalTok{(}\DecValTok{0}\SpecialCharTok{:}\DecValTok{3}\NormalTok{), }\AttributeTok{cex.axis=}\DecValTok{1}\NormalTok{)}
\FunctionTok{axis}\NormalTok{(}\DecValTok{2}\NormalTok{, }\AttributeTok{at =} \FunctionTok{c}\NormalTok{(}\DecValTok{0}\SpecialCharTok{:}\DecValTok{3}\NormalTok{), }\AttributeTok{cex.axis=}\DecValTok{1}\NormalTok{)}
\end{Highlighting}
\end{Shaded}

\includegraphics{trabajo-practico_files/figure-latex/unnamed-chunk-2-1.pdf}

Lo que se observa en el gráfico es que en 1000 repeticiones del ensayo,
se obtienen todos los resultados posibles (todos los del soporte)

\hypertarget{inciso-c}{%
\subparagraph{Inciso c)}\label{inciso-c}}

Hallar la tabla conjunta de frecuencias relativas para cada par (x; y).
Interpretar.

\begin{Shaded}
\begin{Highlighting}[]
\NormalTok{x }\OtherTok{\textless{}{-}} \FunctionTok{c}\NormalTok{(}\FunctionTok{rep}\NormalTok{(}\DecValTok{0}\NormalTok{,}\DecValTok{4}\NormalTok{), }\FunctionTok{rep}\NormalTok{(}\DecValTok{1}\NormalTok{,}\DecValTok{3}\NormalTok{), }\FunctionTok{rep}\NormalTok{(}\DecValTok{2}\NormalTok{,}\DecValTok{2}\NormalTok{), }\FunctionTok{rep}\NormalTok{(}\DecValTok{3}\NormalTok{,}\DecValTok{1}\NormalTok{))}
\NormalTok{y }\OtherTok{\textless{}{-}} \FunctionTok{c}\NormalTok{(}\FunctionTok{seq}\NormalTok{(}\DecValTok{0}\NormalTok{,}\DecValTok{3}\NormalTok{), }\FunctionTok{seq}\NormalTok{(}\DecValTok{0}\NormalTok{,}\DecValTok{2}\NormalTok{), }\FunctionTok{seq}\NormalTok{(}\DecValTok{0}\NormalTok{,}\DecValTok{1}\NormalTok{), }\DecValTok{0}\NormalTok{)}
\NormalTok{frecuencia }\OtherTok{\textless{}{-}} \FunctionTok{rep}\NormalTok{(}\ConstantTok{NA}\NormalTok{, }\FunctionTok{length}\NormalTok{(x))}
\ControlFlowTok{for}\NormalTok{(i }\ControlFlowTok{in} \DecValTok{1}\SpecialCharTok{:}\FunctionTok{length}\NormalTok{(x))\{}
\NormalTok{frecuencia[i] }\OtherTok{\textless{}{-}} \FunctionTok{nrow}\NormalTok{(resultados[resultados}\SpecialCharTok{$}\NormalTok{X}\SpecialCharTok{==}\NormalTok{x[i] }\SpecialCharTok{\&}\NormalTok{ resultados}\SpecialCharTok{$}\NormalTok{Y}\SpecialCharTok{==}\NormalTok{y[i] , ])}

\NormalTok{\}}
\NormalTok{resultado }\OtherTok{\textless{}{-}} \FunctionTok{paste}\NormalTok{(}\FunctionTok{paste}\NormalTok{(x,}\StringTok{"V"}\NormalTok{, }\AttributeTok{sep=}\StringTok{""}\NormalTok{), }\FunctionTok{paste}\NormalTok{(y,}\StringTok{"R"}\NormalTok{, }\AttributeTok{sep=}\StringTok{""}\NormalTok{), }\AttributeTok{sep=}\StringTok{"{-}"}\NormalTok{)}

\NormalTok{tabla\_frecuencias }\OtherTok{\textless{}{-}} \FunctionTok{data.frame}\NormalTok{(x, y, resultado, frecuencia)}

\NormalTok{tabla\_frecuencias}\SpecialCharTok{$}\NormalTok{resultado }\OtherTok{\textless{}{-}} \FunctionTok{as.factor}\NormalTok{(tabla\_frecuencias}\SpecialCharTok{$}\NormalTok{resultado)}
\FunctionTok{print}\NormalTok{(tabla\_frecuencias)}
\end{Highlighting}
\end{Shaded}

\begin{verbatim}
##    x y resultado frecuencia
## 1  0 0     0V-0R         11
## 2  0 1     0V-1R         65
## 3  0 2     0V-2R         83
## 4  0 3     0V-3R          7
## 5  1 0     1V-0R        109
## 6  1 1     1V-1R        305
## 7  1 2     1V-2R         90
## 8  2 0     2V-0R        147
## 9  2 1     2V-1R        149
## 10 3 0     3V-0R         34
\end{verbatim}

\begin{Shaded}
\begin{Highlighting}[]
\FunctionTok{plot}\NormalTok{(tabla\_frecuencias}\SpecialCharTok{$}\NormalTok{resultado, tabla\_frecuencias}\SpecialCharTok{$}\NormalTok{frecuencia, }\AttributeTok{xlab=}\StringTok{"resultado"}\NormalTok{, }\AttributeTok{ylab=}\StringTok{"frecuencia"}\NormalTok{, }\AttributeTok{cex.lab=}\FloatTok{0.01}\NormalTok{)}
\end{Highlighting}
\end{Shaded}

\includegraphics{trabajo-practico_files/figure-latex/unnamed-chunk-4-1.pdf}

Lo que se observa en la tabla de frecuencias (y en el gráfico) es que el
resultado más frecuente es extraer 1 bolilla verde y una roja entre las
3 bolillas extraídas.

Para hacer una mejor interpretación se calcularán las probabilidades de
obtener cada resultado posible, y luego se multiplicará por las
repeticiones (1000) para obtener una frecuencia teórica de resultados.

\begin{Shaded}
\begin{Highlighting}[]
\NormalTok{combinatorio }\OtherTok{\textless{}{-}} \ControlFlowTok{function}\NormalTok{(n,r)\{}
  \ControlFlowTok{if}\NormalTok{ (r}\SpecialCharTok{==}\DecValTok{0}\NormalTok{)\{}
\NormalTok{    resultado }\OtherTok{\textless{}{-}} \DecValTok{1}
\NormalTok{  \} }\ControlFlowTok{else}\NormalTok{ \{}
\NormalTok{      resultado }\OtherTok{\textless{}{-}} \FunctionTok{factorial}\NormalTok{(n) }\SpecialCharTok{/}\NormalTok{ ( }\FunctionTok{factorial}\NormalTok{(n}\SpecialCharTok{{-}}\NormalTok{r)}\SpecialCharTok{*}\FunctionTok{factorial}\NormalTok{(r) )  }
\NormalTok{  \}}
\NormalTok{  resultado}
\NormalTok{\}}


\NormalTok{pxy }\OtherTok{\textless{}{-}} \FunctionTok{rep}\NormalTok{(}\ConstantTok{NA}\NormalTok{, }\DecValTok{10}\NormalTok{)}
\ControlFlowTok{for}\NormalTok{ (i }\ControlFlowTok{in} \DecValTok{1}\SpecialCharTok{:}\DecValTok{10}\NormalTok{)\{}
\NormalTok{  pxy[i] }\OtherTok{\textless{}{-}} \FunctionTok{combinatorio}\NormalTok{(}\DecValTok{4}\NormalTok{, tabla\_frecuencias}\SpecialCharTok{$}\NormalTok{x[i]) }\SpecialCharTok{*} \FunctionTok{combinatorio}\NormalTok{(}\DecValTok{3}\NormalTok{, tabla\_frecuencias}\SpecialCharTok{$}\NormalTok{y[i]) }\SpecialCharTok{*} \FunctionTok{combinatorio}\NormalTok{(}\DecValTok{3}\NormalTok{, }\DecValTok{3}\SpecialCharTok{{-}}\NormalTok{tabla\_frecuencias}\SpecialCharTok{$}\NormalTok{x[i]}\SpecialCharTok{{-}}\NormalTok{tabla\_frecuencias}\SpecialCharTok{$}\NormalTok{y[i]) }\SpecialCharTok{/} \FunctionTok{combinatorio}\NormalTok{(}\DecValTok{10}\NormalTok{,}\DecValTok{3}\NormalTok{)  }
\NormalTok{\}}

\NormalTok{fxy\_teorica }\OtherTok{\textless{}{-}} \DecValTok{1000}\SpecialCharTok{*}\NormalTok{pxy}
\FunctionTok{plot}\NormalTok{(tabla\_frecuencias}\SpecialCharTok{$}\NormalTok{resultado, tabla\_frecuencias}\SpecialCharTok{$}\NormalTok{frecuencia, }\AttributeTok{xlab=}\StringTok{"resultado"}\NormalTok{, }\AttributeTok{ylab=}\StringTok{"frecuencia"}\NormalTok{, }\AttributeTok{cex.lab=}\FloatTok{0.01}\NormalTok{)}
\FunctionTok{points}\NormalTok{(tabla\_frecuencias}\SpecialCharTok{$}\NormalTok{resultado, fxy\_teorica, }\AttributeTok{col=}\StringTok{"red"}\NormalTok{)}
\end{Highlighting}
\end{Shaded}

\includegraphics{trabajo-practico_files/figure-latex/unnamed-chunk-5-1.pdf}

Se observa que los resutados de la simulación y los resultados
``teóricos'' se acercan mucho.

\hypertarget{inciso-d}{%
\subparagraph{Inciso d)}\label{inciso-d}}

Para cada valor observado x, calcular el promedio de los valores de y
correspondientes.

\begin{Shaded}
\begin{Highlighting}[]
\NormalTok{rango\_x }\OtherTok{\textless{}{-}}\NormalTok{ (}\DecValTok{0}\SpecialCharTok{:}\DecValTok{3}\NormalTok{)}
\NormalTok{promedio\_y }\OtherTok{\textless{}{-}} \FunctionTok{c}\NormalTok{(}\FunctionTok{mean}\NormalTok{(resultados[resultados}\SpecialCharTok{$}\NormalTok{X}\SpecialCharTok{==}\DecValTok{0}\NormalTok{ ,]}\SpecialCharTok{$}\NormalTok{Y) , }\FunctionTok{mean}\NormalTok{(resultados[resultados}\SpecialCharTok{$}\NormalTok{X}\SpecialCharTok{==}\DecValTok{1}\NormalTok{ ,]}\SpecialCharTok{$}\NormalTok{Y) , }\FunctionTok{mean}\NormalTok{(resultados[resultados}\SpecialCharTok{$}\NormalTok{X}\SpecialCharTok{==}\DecValTok{2}\NormalTok{ ,]}\SpecialCharTok{$}\NormalTok{Y) , }\FunctionTok{mean}\NormalTok{(resultados[resultados}\SpecialCharTok{$}\NormalTok{X}\SpecialCharTok{==}\DecValTok{3}\NormalTok{ ,]}\SpecialCharTok{$}\NormalTok{Y))}
\FunctionTok{print}\NormalTok{(}\FunctionTok{cbind}\NormalTok{(rango\_x, promedio\_y))}
\end{Highlighting}
\end{Shaded}

\begin{verbatim}
##      rango_x promedio_y
## [1,]       0  1.5180723
## [2,]       1  0.9623016
## [3,]       2  0.5033784
## [4,]       3  0.0000000
\end{verbatim}

\hypertarget{inciso-e}{%
\subparagraph{Inciso e)}\label{inciso-e}}

Graficar los promedios de y en función de los valores observados x,
sobre el gráfico obtenido en el punto b.

\begin{Shaded}
\begin{Highlighting}[]
\FunctionTok{plot}\NormalTok{(resultados}\SpecialCharTok{$}\NormalTok{X, resultados}\SpecialCharTok{$}\NormalTok{Y, }\AttributeTok{xlab=}\StringTok{"X cantidad verdes"}\NormalTok{, }\AttributeTok{ylab=}\StringTok{"Y cantidad rojas"}\NormalTok{, }\AttributeTok{xaxt=}\StringTok{"n"}\NormalTok{, }\AttributeTok{yaxt=}\StringTok{"n"}\NormalTok{, }\AttributeTok{type=}\StringTok{"p"}\NormalTok{)}
\FunctionTok{axis}\NormalTok{(}\DecValTok{1}\NormalTok{, }\AttributeTok{at =} \FunctionTok{c}\NormalTok{(}\DecValTok{0}\SpecialCharTok{:}\DecValTok{3}\NormalTok{), }\AttributeTok{cex.axis=}\DecValTok{1}\NormalTok{)}
\FunctionTok{axis}\NormalTok{(}\DecValTok{2}\NormalTok{, }\AttributeTok{at =} \FunctionTok{c}\NormalTok{(}\DecValTok{0}\SpecialCharTok{:}\DecValTok{3}\NormalTok{), }\AttributeTok{cex.axis=}\DecValTok{1}\NormalTok{)}
\FunctionTok{points}\NormalTok{(rango\_x, promedio\_y, }\AttributeTok{col=}\StringTok{"red"}\NormalTok{)}
\end{Highlighting}
\end{Shaded}

\includegraphics{trabajo-practico_files/figure-latex/unnamed-chunk-7-1.pdf}

\hypertarget{inciso-f}{%
\subparagraph{Inciso f)}\label{inciso-f}}

Superponer en el gráfico anterior la función de regresión \(\phi(x)\) =
E{[}Y\textbar X = x{]}. Concluir a partir de lo observado.

La función de probabilidad conjunta se obtiene por conteo de casos
favorables sobre casos totales:

\(P_{XY}(x,y)= \frac{C_{3,y}.C_{4,x}.C_{3,3-x-y}}{C_{10,3}}\)

Cada una de las marginales tiene distribusión hipergeométrica, por lo
que:

\(P_{X}(x)= \frac{C_{4,x}.C_{6,3-x}}{C_{10,3}}\)

\(P_{Y}(y)= \frac{C_{3,y}.C_{7,3-y}}{C_{10,3}}\)

Por lo tanto, se obtiene la condicional:

\(P_{Y|X}(y)= \frac{P_{XY}(x,y)}{P_{X}(x)}\)

\(P_{Y|X}(y)= \frac{C_{3,y}.C_{3,3-y-x}}{C_{6,3-x}}\)

Y a partir de la función de probabilidad conjunta, se obtiene la función
de regresión. Para cada X del soporte:

\(E_{[Y|X=x]}=\phi(x)= \Sigma y.P_{Y|X}(y)\)

\begin{Shaded}
\begin{Highlighting}[]
\NormalTok{PYdadoX }\OtherTok{\textless{}{-}} \FunctionTok{data.frame}\NormalTok{(}\AttributeTok{X=}\FunctionTok{rep}\NormalTok{(}\ConstantTok{NA}\NormalTok{,}\DecValTok{10}\NormalTok{), }\AttributeTok{Y=}\FunctionTok{rep}\NormalTok{(}\ConstantTok{NA}\NormalTok{,}\DecValTok{10}\NormalTok{), }\AttributeTok{P\_YdadoX=}\FunctionTok{rep}\NormalTok{(}\ConstantTok{NA}\NormalTok{,}\DecValTok{10}\NormalTok{))}
\NormalTok{contador }\OtherTok{\textless{}{-}} \DecValTok{1}
\ControlFlowTok{for}\NormalTok{ (i }\ControlFlowTok{in} \DecValTok{0}\SpecialCharTok{:}\DecValTok{3}\NormalTok{)\{}
  \ControlFlowTok{for}\NormalTok{(j }\ControlFlowTok{in} \DecValTok{0}\SpecialCharTok{:}\NormalTok{(}\DecValTok{3}\SpecialCharTok{{-}}\NormalTok{i))\{}
\NormalTok{    PYdadoX}\SpecialCharTok{$}\NormalTok{X[contador] }\OtherTok{\textless{}{-}}\NormalTok{ i}
\NormalTok{    PYdadoX}\SpecialCharTok{$}\NormalTok{Y[contador] }\OtherTok{\textless{}{-}}\NormalTok{ j}
\NormalTok{    PYdadoX}\SpecialCharTok{$}\NormalTok{P\_YdadoX[contador] }\OtherTok{\textless{}{-}} \FunctionTok{combinatorio}\NormalTok{(}\DecValTok{3}\NormalTok{, j)}\SpecialCharTok{*}\FunctionTok{combinatorio}\NormalTok{(}\DecValTok{3}\NormalTok{,}\DecValTok{3}\SpecialCharTok{{-}}\NormalTok{i}\SpecialCharTok{{-}}\NormalTok{j) }\SpecialCharTok{/} \FunctionTok{combinatorio}\NormalTok{(}\DecValTok{6}\NormalTok{,}\DecValTok{3}\SpecialCharTok{{-}}\NormalTok{i)}
\NormalTok{    contador }\OtherTok{\textless{}{-}}\NormalTok{ contador}\SpecialCharTok{+}\DecValTok{1}
\NormalTok{  \}}
\NormalTok{\}}

\NormalTok{EYdadoX }\OtherTok{\textless{}{-}} \FunctionTok{data.frame}\NormalTok{(}\AttributeTok{X=}\FunctionTok{seq}\NormalTok{(}\DecValTok{0}\NormalTok{,}\DecValTok{3}\NormalTok{) , }\AttributeTok{E\_YdadoX=}\FunctionTok{rep}\NormalTok{(}\DecValTok{0}\NormalTok{, }\DecValTok{4}\NormalTok{))}
\NormalTok{contador }\OtherTok{\textless{}{-}} \DecValTok{1}

\ControlFlowTok{for}\NormalTok{ (i }\ControlFlowTok{in} \DecValTok{0}\SpecialCharTok{:} \DecValTok{3}\NormalTok{ )\{}
  \ControlFlowTok{for}\NormalTok{(j }\ControlFlowTok{in} \DecValTok{0}\SpecialCharTok{:}\NormalTok{(}\DecValTok{3}\SpecialCharTok{{-}}\NormalTok{i))\{}
\NormalTok{    EYdadoX}\SpecialCharTok{$}\NormalTok{E\_YdadoX[i}\SpecialCharTok{+}\DecValTok{1}\NormalTok{] }\OtherTok{\textless{}{-}}\NormalTok{ EYdadoX}\SpecialCharTok{$}\NormalTok{E\_YdadoX[i}\SpecialCharTok{+}\DecValTok{1}\NormalTok{] }\SpecialCharTok{+}\NormalTok{ j}\SpecialCharTok{*}\NormalTok{PYdadoX}\SpecialCharTok{$}\NormalTok{P\_YdadoX[contador]}
\NormalTok{    contador }\OtherTok{\textless{}{-}}\NormalTok{ contador}\SpecialCharTok{+}\DecValTok{1}
\NormalTok{  \}}
\NormalTok{\}}

\FunctionTok{plot}\NormalTok{(resultados}\SpecialCharTok{$}\NormalTok{X, resultados}\SpecialCharTok{$}\NormalTok{Y, }\AttributeTok{xlab=}\StringTok{"X cantidad verdes"}\NormalTok{, }\AttributeTok{ylab=}\StringTok{"Y cantidad rojas"}\NormalTok{, }\AttributeTok{xaxt=}\StringTok{"n"}\NormalTok{, }\AttributeTok{yaxt=}\StringTok{"n"}\NormalTok{, }\AttributeTok{type=}\StringTok{"p"}\NormalTok{)}
\FunctionTok{axis}\NormalTok{(}\DecValTok{1}\NormalTok{, }\AttributeTok{at =} \FunctionTok{c}\NormalTok{(}\DecValTok{0}\SpecialCharTok{:}\DecValTok{3}\NormalTok{), }\AttributeTok{cex.axis=}\DecValTok{1}\NormalTok{)}
\FunctionTok{axis}\NormalTok{(}\DecValTok{2}\NormalTok{, }\AttributeTok{at =} \FunctionTok{c}\NormalTok{(}\DecValTok{0}\SpecialCharTok{:}\DecValTok{3}\NormalTok{), }\AttributeTok{cex.axis=}\DecValTok{1}\NormalTok{)}
\FunctionTok{points}\NormalTok{(rango\_x, promedio\_y, }\AttributeTok{col=}\StringTok{"red"}\NormalTok{)}
\FunctionTok{points}\NormalTok{(EYdadoX}\SpecialCharTok{$}\NormalTok{X, EYdadoX}\SpecialCharTok{$}\NormalTok{E\_YdadoX, }\AttributeTok{col=}\StringTok{"green"}\NormalTok{)}
\end{Highlighting}
\end{Shaded}

\includegraphics{trabajo-practico_files/figure-latex/unnamed-chunk-8-1.pdf}

Se observa que los puntos obtenidos con la función de regresión
coinciden casi exactamente con los promedios de Y de la simulación.

\hypertarget{ejercicio-2}{%
\subsubsection{Ejercicio 2)}\label{ejercicio-2}}

Realizar 1000 simulaciones del vector aleatorio (X; Y ), cuya densidad
conjunta es de la forma:

\(f_{XY}(x,y)= \frac{1}{2x+1} e^{-2x-\frac{y}{4x+2}}.I(x>0, y>0)\)

\hypertarget{inciso-a-1}{%
\subparagraph{Inciso a)}\label{inciso-a-1}}

Gracar x vs.~y.

\begin{Shaded}
\begin{Highlighting}[]
\FunctionTok{set.seed}\NormalTok{(}\DecValTok{17}\NormalTok{)}
\NormalTok{repeticion }\OtherTok{\textless{}{-}} \FunctionTok{seq}\NormalTok{(}\DecValTok{1}\NormalTok{,}\DecValTok{1000}\NormalTok{)}
\NormalTok{X }\OtherTok{\textless{}{-}} \FunctionTok{rexp}\NormalTok{(}\DecValTok{1000}\NormalTok{, }\AttributeTok{rate=}\DecValTok{2}\NormalTok{)}
\NormalTok{Y }\OtherTok{\textless{}{-}} \FunctionTok{rep}\NormalTok{(}\ConstantTok{NA}\NormalTok{, }\FunctionTok{length}\NormalTok{(X))}
\ControlFlowTok{for}\NormalTok{(i }\ControlFlowTok{in} \DecValTok{1}\SpecialCharTok{:}\FunctionTok{length}\NormalTok{(Y))\{}
\NormalTok{  Y[i] }\OtherTok{\textless{}{-}} \FunctionTok{rexp}\NormalTok{(}\DecValTok{1}\NormalTok{, }\AttributeTok{rate=} \DecValTok{1}\SpecialCharTok{/}\NormalTok{(}\DecValTok{4}\SpecialCharTok{*}\NormalTok{X[i]}\SpecialCharTok{+}\DecValTok{2}\NormalTok{) )}
\NormalTok{\}}

\FunctionTok{plot}\NormalTok{(X,Y)}
\end{Highlighting}
\end{Shaded}

\includegraphics{trabajo-practico_files/figure-latex/unnamed-chunk-9-1.pdf}

A partir de la función de distribución conjunta, obtenemos la función de
distribución marginal

\(f_{XY}(x,y)= \frac{1}{2x+1} e^{-2x-\frac{y}{4x+2}}\)

\(f_{X}(x)= 2e^{-2x}\)

Que resulta una exponencial de parámetro:

\(\lambda= 2\)

Luego:

\(f_{Y|X}(y) = \frac{1}{4x+2}e^{-\frac{y}{4x+2}}\)

Que resulta una exponencial de parámetro:

\(\lambda= \frac{1}{4x+2}\)

Para hacer la simulación, se simuló X con la distribución dada. Luego
para cada X obtenido en la simulación, se simuló Y dado X.

\hypertarget{inciso-b-1}{%
\subparagraph{Inciso b)}\label{inciso-b-1}}

Para cada valor observado x, tomar una ventana de (x-h; x+h), y calcular
el promedio de los valores de Y para todas las observaciones que caen
dentro de dicho intervalo. Elegir el valor de h que crea adecuado.
Justicar.

\begin{Shaded}
\begin{Highlighting}[]
\NormalTok{XY }\OtherTok{\textless{}{-}} \FunctionTok{data.frame}\NormalTok{(X,Y)}
\NormalTok{h }\OtherTok{\textless{}{-}} \FunctionTok{seq}\NormalTok{(}\FloatTok{0.01}\NormalTok{, }\DecValTok{2}\NormalTok{ , }\FloatTok{0.01}\NormalTok{)}

\ControlFlowTok{for}\NormalTok{ (j }\ControlFlowTok{in} \DecValTok{1}\SpecialCharTok{:}\FunctionTok{length}\NormalTok{(h))\{}
\NormalTok{  Y\_promedio }\OtherTok{\textless{}{-}} \FunctionTok{rep}\NormalTok{(}\ConstantTok{NA}\NormalTok{, }\FunctionTok{nrow}\NormalTok{(XY))}
  \ControlFlowTok{for}\NormalTok{(i }\ControlFlowTok{in} \DecValTok{1}\SpecialCharTok{:}\FunctionTok{nrow}\NormalTok{(XY))\{}
    \ControlFlowTok{if}\NormalTok{( }\FunctionTok{nrow}\NormalTok{( XY[}\FunctionTok{abs}\NormalTok{(XY}\SpecialCharTok{$}\NormalTok{X[i] }\SpecialCharTok{{-}}\NormalTok{ XY}\SpecialCharTok{$}\NormalTok{X)}\SpecialCharTok{\textless{}}\NormalTok{h[j] ,]) }\SpecialCharTok{\textgreater{}}\DecValTok{1}\NormalTok{ )\{}
      \DocumentationTok{\#\#vector con el promedio de los valores cercanos, sin usar el Y en el punto}
\NormalTok{      Y\_promedio[i] }\OtherTok{\textless{}{-}}\NormalTok{  ( }\FunctionTok{sum}\NormalTok{( XY[ }\FunctionTok{abs}\NormalTok{(XY}\SpecialCharTok{$}\NormalTok{X[i] }\SpecialCharTok{{-}}\NormalTok{ XY}\SpecialCharTok{$}\NormalTok{X)}\SpecialCharTok{\textless{}}\NormalTok{h[j] ,]}\SpecialCharTok{$}\NormalTok{Y) }\SpecialCharTok{{-}}\NormalTok{ XY}\SpecialCharTok{$}\NormalTok{Y[i] ) }\SpecialCharTok{/}\NormalTok{ (}\FunctionTok{nrow}\NormalTok{( XY[}\FunctionTok{abs}\NormalTok{(XY}\SpecialCharTok{$}\NormalTok{X[i] }\SpecialCharTok{{-}}\NormalTok{ XY}\SpecialCharTok{$}\NormalTok{X)}\SpecialCharTok{\textless{}}\NormalTok{h[j] ,])}\SpecialCharTok{{-}}\DecValTok{1}\NormalTok{)  }
\NormalTok{    \} }\ControlFlowTok{else}\NormalTok{ \{}
\NormalTok{      Y\_promedio[i] }\OtherTok{\textless{}{-}}\NormalTok{ XY}\SpecialCharTok{$}\NormalTok{Y[i]}
\NormalTok{    \}}
\NormalTok{  \}}
\NormalTok{  XY }\OtherTok{\textless{}{-}} \FunctionTok{cbind}\NormalTok{(XY, Y\_promedio)}
\NormalTok{\}}
\DocumentationTok{\#\#Error cuadrático medio por convalidación cruzada}
\NormalTok{ECM\_CV }\OtherTok{\textless{}{-}} \FunctionTok{rep}\NormalTok{(}\ConstantTok{NA}\NormalTok{, }\FunctionTok{length}\NormalTok{(h))}
\ControlFlowTok{for}\NormalTok{(i }\ControlFlowTok{in} \DecValTok{1}\SpecialCharTok{:}\FunctionTok{length}\NormalTok{(h))\{}
\NormalTok{  ECM\_CV[i] }\OtherTok{\textless{}{-}} \FunctionTok{mean}\NormalTok{((XY}\SpecialCharTok{$}\NormalTok{Y}\SpecialCharTok{{-}}\NormalTok{XY[,i}\SpecialCharTok{+}\DecValTok{2}\NormalTok{])}\SpecialCharTok{**}\DecValTok{2}\NormalTok{)}
\NormalTok{\}}

\NormalTok{h\_opt }\OtherTok{\textless{}{-}}\NormalTok{ h[}\FunctionTok{which}\NormalTok{(ECM\_CV}\SpecialCharTok{==}\FunctionTok{min}\NormalTok{(ECM\_CV[}\SpecialCharTok{{-}}\DecValTok{3}\NormalTok{]))]}
\FunctionTok{plot}\NormalTok{(h, ECM\_CV)}
\end{Highlighting}
\end{Shaded}

\includegraphics{trabajo-practico_files/figure-latex/unnamed-chunk-10-1.pdf}

Para obtener la ventanan óptima se calculó el erros cuadrático medio
para distintas h usando el método de convalidación cruzada. En cada X se
obtuvo el promedio de los Y incluídos en la ventana h, dejando afuera el
Y correspondiente al X del centro de la ventana (método ``Leave one
out'').

Si bien del gráfico se desprende que con los valores más chicos de h
podría alcanzarse el mínimo, se decidió dejar afuera del análisis los h
tan chicos por riesgo de ``over-fitting'', resultando el h óptimo = 0,22

\hypertarget{inciso-c-1}{%
\subparagraph{Inciso c)}\label{inciso-c-1}}

Graficar los promedios de y en función de los valores observados x,
sobre el gráfico obtenido en el punto a.

\begin{Shaded}
\begin{Highlighting}[]
\NormalTok{Y\_promedio\_optimo }\OtherTok{\textless{}{-}} \FunctionTok{rep}\NormalTok{(}\ConstantTok{NA}\NormalTok{, }\FunctionTok{length}\NormalTok{(Y))}
\ControlFlowTok{for}\NormalTok{(i }\ControlFlowTok{in} \DecValTok{1}\SpecialCharTok{:}\FunctionTok{length}\NormalTok{(Y))\{}
\NormalTok{  Y\_promedio\_optimo[i] }\OtherTok{\textless{}{-}} \FunctionTok{mean}\NormalTok{( XY[ }\FunctionTok{abs}\NormalTok{(XY}\SpecialCharTok{$}\NormalTok{X[i] }\SpecialCharTok{{-}}\NormalTok{ XY}\SpecialCharTok{$}\NormalTok{X)}\SpecialCharTok{\textless{}}\NormalTok{h\_opt ,]}\SpecialCharTok{$}\NormalTok{Y)}
\NormalTok{\}}

\FunctionTok{plot}\NormalTok{(X,Y)}
\FunctionTok{points}\NormalTok{(X, Y\_promedio\_optimo, }\AttributeTok{col=}\StringTok{"green"}\NormalTok{)}
\end{Highlighting}
\end{Shaded}

\includegraphics{trabajo-practico_files/figure-latex/unnamed-chunk-11-1.pdf}

\hypertarget{inciso-d-1}{%
\subparagraph{Inciso d)}\label{inciso-d-1}}

Superponer en el gráfico anterior la función de regresión \(\phi(x)\) =
E{[}Y\textbar X = x{]}.

\begin{Shaded}
\begin{Highlighting}[]
\NormalTok{E\_YdadoX }\OtherTok{\textless{}{-}} \DecValTok{4}\SpecialCharTok{*}\NormalTok{X}\SpecialCharTok{+}\DecValTok{2}

\FunctionTok{plot}\NormalTok{(X,Y)}
\FunctionTok{points}\NormalTok{(X, Y\_promedio\_optimo, }\AttributeTok{col=}\StringTok{"green"}\NormalTok{)}
\FunctionTok{points}\NormalTok{(X, E\_YdadoX, }\AttributeTok{col=}\StringTok{"red"}\NormalTok{, }\AttributeTok{type=}\StringTok{"l"}\NormalTok{)}
\end{Highlighting}
\end{Shaded}

\includegraphics{trabajo-practico_files/figure-latex/unnamed-chunk-12-1.pdf}

\end{document}
